\documentclass[listings]{labreport}
\subject{Тестирование программного обеспечения}
\titleparts{Лабораторная работа №1}{Вариант 244}
\students{Лабушев Тимофей}

\begin{document}

\maketitlepage

\section*{Задание}

\begin{enumerate}
  \item Для функции $sin(x)$ провести модульное тестирование разложения функции в степенной ряд. Выбрать достаточное тестовое покрытие.
  \item Провести модульное тестирование программного модуля для работы с биномиальной кучей
    (Logical Representation,\newline\verb|http://www.cs.usfca.edu/~galles/visualization/BinomialQueue.html|)
  \item Сформировать доменную модель для заданного текста:
\begin{verbatim}
Зажужжал мотор. Тоненький свист перерос в рев воздуха, вырывающегося в черную
пустоту, усеянную невероятно яркими светящимися точками. Форд и Артур вылетели
в открытый космос, как конфетти из хлопушки. Глава 8
\end{verbatim}
    Разработать тестовое покрытие для данной доменной модели.
\end{enumerate}

\section*{Инструменты тестирования}

Для выполнения лабораторной работы использовалась библиотека JUnit 5 (модуль Jupiter).
Отличия от JUnit 4, важные в контексте выполнения данного задания — новые возможности,
описанные в выводе, а также синтаксические изменения, коснувшиеся аннотаций
тестовых методов и классов.

\section*{Исходный код}

\verb|https://github.com/timlathy/itmo-fourth-year/tree/master/Software-Testing-7th-Term/Lab1|

\section*{Выводы}

В ходе работы было рассмотрено модульное тестирование программных модулей
при помощи библиотеки JUnit 5 (модуль Jupiter). При написании тестов были изучены
такие нововведения последней версии, как группировка тестов аннотацией \verb|@Nested|,
работа с исключениями используя \verb|assertThrows|, упрощенное написание
параметризованных тестов при помощи аннотации \verb|@ParameterizedTest|.

\end{document}
