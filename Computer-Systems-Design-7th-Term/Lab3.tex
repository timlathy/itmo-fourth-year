\documentclass[listings]{labreport}
\subject{Проектирование вычислительных систем}
\titleparts{Лабораторная работа №3}{Разработка инструмента проектирования архитектурного уровня}
\students{Лабушев Тимофей}

\usepackage{float}
\usepackage[normalem]{ulem}
\usepackage{booktabs}
\usepackage{enumitem}
\usepackage{array}
\usepackage{pdflscape}

\newcolumntype{L}[1]{>{\raggedright\let\newline\\\arraybackslash\hspace{0pt}}m{#1}}

\begin{document}

\maketitlepage

\section*{Формулировка решаемой задачи}

В ходе выполнения предыдущей работы было обнаружено, что \textit{диаграмма потоков данных}
недостаточно подробно описывает преобразование данных в системе. В частности, она
не позволяет эффективно отразить ошибки, возникающие на различных этапах, что
скрывает реальную сложность системы.

Обработка ошибок может быть отражена \textit{диаграммой деятельности}, однако ее сложность
существенно увеличивается с ростом количества рассматриваемых точек и причин отказа. 

В работе рассматривается инструмент архитектурного проектирования, который
расширяет \textit{диаграммы потоков данных} элементами для рассмотрения ошибок.

Принимается следующая классификация ошибок:
\begin{itemize}[noitemsep,topsep=0em]
\item \textit{восстановимые во время исполнения} — неверные входные данные,
  временная недоступность внешних сервисов и т.д;
\item \textit{невосстановимые} — ошибки программиста, такие как разыменование нулевого указателя и
  выход за границы массива.
\end{itemize}

Предлагаемый инструмент нацелен на планирование \textit{восстановимых ошибок} до реализации системы,
поскольку их обработка может повлиять на другие архитектурные решения. Составление диаграммы требует
тщательного обдумывания различных видов отказа, а ее обсуждение позволяет выявить неочевидные
исключительные ситуации еще до написания кода.

\textit{Невосстановимые} ошибки не рассматриваются, поскольку их возникновение не зависит от архитектуры
системы. Для их предотвращения применимы статический анализ, тестирование, другие процессы разработки ПО.

\section*{Описание инструмента архитектурного проектирования}

\renewcommand{\arraystretch}{1.5}
\noindent
\begin{tabular}{p{3.2cm}p{13.5cm}} 
  \toprule
  \textbf{Описание}
  & \uline{Диаграмма потоков результатов} отображает преобразование данных
  в системе с учетом восстановимых ошибок.
  \\
  \textbf{Элементы}
  & \uline{Фильтр}. Принимает на вход данные, выставляет результат на \textit{выход данных}
  или \textit{выход ошибки}.
  \newline\uline{Артефакт}. Набор входных или выходных данных из одного источника.
  \\
  \textbf{Отношения}
  & \uline{Связь данных} (сплошная линия) сопоставляет входной артефакт или \textit{выход данных}
  предшествующего фильтра с входом последующего фильтра или выходным артефактом.
  \newline\uline{Связь ошибки} (пунктирная линия) сопоставляет \textit{выход ошибки} фильтра
  с входом последующего фильтра или артефактом ошибки.
  \\
  \textbf{Ограничения}
  & \begin{minipage}[t]{\linewidth}
    \begin{itemize}[leftmargin=*,noitemsep,nosep]
    \item Каждый фильтр содержит один или несколько входов, один \textit{выход данных} и один
      \textit{выход ошибки}.
    \item Результатом работы фильтра не могут быть данные и ошибка одновременно.
    \item Типы передаваемых между фильтрами данных не указываются, поскольку их несовместимость — \textit{невосстановимая} ошибка.
    \item Следует избегать циклов в диаграмме, так как это ухудшает читаемость и отвлекает от сути.
    \item Не стоит излишне детализировать, что приведет к ошибке, к примеру, чтения файла
      (отсутствие прав, неверный путь и т.д.), достаточно того, что подобная ошибка может возникнуть.
    \end{itemize}
    \end{minipage}
  \\
  \textbf{Применение}
  & \begin{minipage}[t]{\linewidth}
    \begin{itemize}[leftmargin=*,noitemsep,nosep]
    \item Повышение отказоустойчивости системы.
    \item Обнаружение неочевидных исключительных ситуаций до начала реализации.
    \item Избежание некорректных архитектурных решений вследствие недостаточности информации о системе.
    \item Наглядное представление валидности данных на каждом из этапов преобразования.
    \end{itemize}
    \end{minipage}
    \vspace*{0.1em}
  \\\bottomrule
\end{tabular}

\section*{Пример использования инструмента архитектурного проектирования}

На рисунках \ref{fig:disasm-ext} и \ref{fig:disasm-int} представлено сравнение
способов получения промежуточного представления исполняемого файла для статического анализа,
которое обсуждалось в предыдущей работе.

\begin{landscape}
\vspace*{\fill}
\begin{figure}[H]
\centering
\includegraphics[width=1.5\textwidth]{rfd-disasm-ext.png}
\caption{\small{Обеспечение совместимости с различными наборами команд.
    Использование внешнего дизассемблера.}}
\label{fig:disasm-ext}
\end{figure}
\vfill
\end{landscape}

\begin{landscape}
\vspace*{\fill}
\begin{figure}[H]
\centering
\includegraphics[width=1.2\textwidth]{rfd-disasm-int.png}
\caption{\small{Обеспечение совместимости с различными наборами команд.
   Интеграция дизассемблера.}}
\label{fig:disasm-int}
\end{figure}
\vfill
\end{landscape}

\section*{Критерии оценки инструмента архитектурного проектирования}

В работе выделяются следующие критерии оценки:
\begin{enumerate}[noitemsep,topsep=0em]
\item \textbf{Простота восприятия}. Противоположна нагруженности информацией, будет выше
  для специализированных диаграмм.
\item \textbf{Простота создания и изменения}. Отражает возможность использования инструмента в
  процессе живого обсуждения.
\item \textbf{Гибкость}. Оценивает способность инструмента отобразить систему
  на различных уровнях детализации.
\item \textbf{Общеприменимость}. Оценивает способность инструмента отобразить различные виды систем.
\item \textbf{Полнота представления системы}. Соответствует объему охватываемых инструментом аспектов системы.
\item \textbf{Повышение отказоустойчивости системы}. Показывает, насколько инструмент соответствует задаче
  проектирования с учетом возможных ошибок во время работы системы.
\end{enumerate}

\section*{Сравнительный анализ предлагаемого и альтернативных инструментов проектирования}

Для сравнительного анализа выбраны следующие инструменты проектирования:
\begin{itemize}[noitemsep,topsep=0em]
\item \textit{Диаграмма потоков данных}, поскольку лежит в основе предлагаемого инструмента проектирования;
\item \textit{Диаграмма деятельности}, поскольку является более гибкой и может в общем случае представить
  как ДПД, так и ДПР.
\end{itemize}

Оценивание производится по шкале от 1 до 5, где 5 показывает наилучшее соответствие критерию.

\begin{tabular}{L{5.5cm} L{3cm} L{3cm} L{3cm}}
  \toprule
  Критерий & Диаграмма потоков данных & Диаграмма потоков результатов & Диаграмма деятельности \\\hline
  Простота восприятия & 4 & \textbf{5} & 3 \\\hline
  Простота создания и изменения & \textbf{4} & \textbf{4} & 2 \\\hline
  Гибкость & 3 & 2 & \textbf{5} \\\hline
  Общеприменимость & 3 & 2 & \textbf{4} \\\hline
  Полнота представления системы & 3 & 3 & \textbf{5} \\\hline
  Повышение отказоустойчивости системы & 3 & \textbf{5} & 2 \\
  \bottomrule
\end{tabular}

\vspace{1em}

Визуализируем результаты при помощи лепестковой диаграммы, чтобы облегчить сравнение:

\begin{figure}[H]
\centering
\includegraphics[width=0.9\textwidth]{lab3-compchart.pdf}
\caption{\small{Результаты сравнительного анализа инструментов проектирования}}
\label{fig:compchart}
\end{figure}

\section*{Вывод}

В ходе выполнения лабораторной работы на основе \textit{диаграмм потоков данных}
был создан новый инструмент проектирования, позволяющий улучшить отказоустойчивость
систем за счет раннего планирования \textit{восстановимых} ошибок.

Сравнительный анализ позволил выявить как сильные стороны, так и ограничения
представленного инструмента. \textit{Диаграмма потоков результатов} обращает
внимание проектировщика на виды отказа на различных этапах работы системы.
Представление лишено деталей, что облегчает создание и изменение диаграммы
(как в цифровом виде, так и, например, на белой доске в конференц-зале).
Как следствие, поощряется обсуждение, сосредоточенное на выявлении и планировании
отказов.

Тем не менее, инструмент довольно ограничен и узконаправлен. В отличие от диаграммы деятельности,
ДПР подходит лишь там, где можно выделить цепь преобразования данных
(ориентированный ациклический граф). ДПР неудобна и для первоначального проектирования
такого графа, когда определяется \textit{happy path} и типы соединений — для
подобных задач следует использовать диаграммы потоков данных.

\end{document}
